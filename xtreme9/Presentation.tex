% Copyright (c) 2015 Benito Palacios Sánchez - All Rights Reserved.
% Esta obra está licenciada bajo la Licencia Creative Commons Atribución 4.0
% Internacional. Para ver una copia de esta licencia, visita
% http://creativecommons.org/licenses/by/4.0/.

% Template
\documentclass[usenames,dvipsnames]{beamer}

% Notes. Uncomment to view the notes.
%\setbeameroption{show notes}
%\setbeamertemplate{note page}[plain]    % Remove the note page style

% Font
\usepackage[T1]{fontenc}        % Output font
\usepackage[utf8]{inputenc}	% Input encoding
\usepackage[spanish]{babel}    	% For Spanish texts
\usepackage{FiraSans}

% Theme
\usetheme{Darmstadt}
\usecolortheme{whale}

% Other packages
\usepackage{xcolor}             % For color in text
\usepackage{url}                % For links
\usepackage{pifont}             % For tick symbol
\usepackage{graphicx}           % For graphics
\usepackage{epstopdf}			% For EPS graphics in Windows
\usepackage{multimedia}         % For media
\usepackage{verbatim}           % For non-parsed text blocks
\usepackage{listings}           % For blocks of code
\lstset{language=[Sharp]C,basicstyle=\scriptsize\ttfamily, keywordstyle=\scriptsize\color{blue}\ttfamily}

% My package
\usepackage{Layout}

% Information about author and document
\title{IEEEXtreme 9.0}
\date[Septiembre de 2015]{\today}
\author{Benito Palacios Sánchez}
\authortitle{}
\authoremail{benito.palaciossanchez.es@ieee.org}
\institute[IEEE SB UGR]{IEEE Student Branch of Granada}
\titlelogo{imgs/logo.png}

% Add a little logo in the corner of the slides
\pgfdeclareimage[height=0.5cm]{logo-mini}{imgs/logo_mini.png}
\logo{\pgfuseimage{logo-mini}}

\begin{document}

% Title page
\begin{frame}[plain]
    \titlepage
\end{frame}


\section{IEEEXtreme 9.0}
\subsection{IEEEXtre... What?}

% Timelapse video
\begin{frame}
	\movie[width=\textwidth,
	       externalviewer]
	       {\includegraphics[width=\textwidth]{videos/Timelapse.png}}
	       {videos/Timelapse.mp4}
\end{frame}

\begin{frame}
	\includegraphics[width=\textwidth,keepaspectratio]{imgs/q1.png}

  	\begin{wideitemize}
  		\item \textbf{24 horas} de programación en el lenguaje que quieras.
  		\item 23 Problemas de lógica, programación y matemáticas.
  		\item Compite con más de 1750 equipos de todo el mundo.
  	\end{wideitemize}
\end{frame}

\section[IEEE SB UGR]{Rama estudiantil de IEEE para la UGR}
\subsection{IEEE}
\begin{frame}
\end{frame}

\subsection{Ramas estudiantiles}
\begin{frame}
\end{frame}

\subsection{Actividades}
\begin{frame}
\end{frame}

\end{document}
