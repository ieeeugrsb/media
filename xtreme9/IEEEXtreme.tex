% Copyright (c) 2015 Benito Palacios Sánchez - All Rights Reserved.
% Esta obra está licenciada bajo la Licencia Creative Commons Atribución 4.0
% Internacional. Para ver una copia de esta licencia, visita
% http://creativecommons.org/licenses/by/4.0/.

\section{IEEEXtreme 9.0}

\subsection{IEEEXtre... What?}
% Timelapse video
\begin{frame}
	\movie[width=\textwidth,
	       externalviewer]
	       {\includegraphics[width=\textwidth]{videos/Timelapse.png}}
	       {videos/Timelapse.mp4}
\end{frame}

\begin{frame}{24 horas non-stop}
    \begin{center}
        \includegraphics[width=0.85\textwidth,keepaspectratio]{imgs/q1.png}
    \end{center}
    
    \begin{columns}
    \begin{column}{0.30\textwidth}
        \footnotesize
        \textbf{Entrada:}
        
        \texttt{1000 999 10}
        
        \texttt{210 548 889 7...} \\~
        
        \textbf{Salida esperada:}
        
        \texttt{12}
    \end{column}
    \begin{column}{0.80\textwidth}
        \begin{itemize}
            \small
            \item \textbf{24 horas} de programación por equipos.
            \item 23 problemas de lógica, programación y matemáticas.
            \item Compite con más de 1750 equipos de todo el mundo.
        \end{itemize}
    \end{column}
    \end{columns}
\end{frame}

\subsection{Equipos}
\begin{frame}{Equipos}
    \begin{center}
		\large Equipos de 1 a 3 estudiantes miembros de IEEE.
	\end{center}
    
    \begin{columns}\begin{column}{0.60\textwidth}
        \includegraphics[width=\textwidth]{imgs/team.jpg}
    \end{column}\begin{column}{0.40\textwidth}
        \includegraphics[width=\textwidth]{imgs/mug0.jpg}
    \end{column}\end{columns}
\end{frame}

\subsection{Lenguajes permitidos}
\begin{frame}{Elige tu lenguaje}
\ctable[
    width = \textwidth,
    doinside = \footnotesize
]{llcl}{
}{                                                                             \FL
    Lenguaje & Versión           & Máx. seg. & Extras                          \ML
    C        & 4.9.2 C99 std     & 2         & Math, -lm, json                 \NN
    C++      & 4.9.2 C++11 std   & 2         & Math, -lm, json                 \NN
    C\#      & Mono 3.2 .NET 4.0 & 3         & newtonsoft                      \NN
    Python   & 2.7.6             & 10        &                                 \NN
    Python 3 & 3.4.0             & 10        &                                 \NN
    Java     & 1.7.0\_55         & 4         & json-simple                     \NN
    Java 8   & 1.8.0\_05         & 4         & json-simple                     \NN
    PHP      & 5.5.9             & 9         &                                 \NN
    Perl     & 5.18.2            & 9         & json                            \NN
    Ruby     & 2.0               & 10        &                                 \NN
    Obj-C    & Obj-C 2.0         & 2         & gnustep-libobjc2, foundation,...\NN
    Haskell  & Ghc 7.8.4         & 5         & logict lens pipes mwc-random,...\LL
}
\end{frame}

\begin{frame}{Elige tu lenguaje}
\ctable[
    width = \textwidth,
    doinside = \footnotesize
]{llcl}{
}{                                                                             \FL
    Lenguaje & Versión           & Máx. seg. & Extras                          \ML
    Clojure  & 1.6.0             & 8         &                                 \NN
    Scala    & 2.11.0            & 7         &                                 \NN
    Common Lisp & SBCL 1.2.3     & 12        &                                 \NN
    Lua      & 5.2.3             & 12        &                                 \NN
    Erlang   & 6.3               & 12        &                                 \NN
    JS       & Node 0.10.28      & 10        &                                 \NN
    Go       & 1.4               & 4         &                                 \NN
    Groovy   & 1.8.6             & 5         & JVM: 1.7.0\_55                  \NN
    OCaml    & Ocamlopt, 4.01.0  & 3         & Jane Street OCaml core lib      \NN
    F\#      & F\# 3.0, Mono 3.2 & 4         &                                 \NN
    VB.NET   & Mono 3.2 .NET 4.0 & 5         &                                 \NN
    LOLCODE  & 1.2 con Ici 0.10.5& 5         &                                 \LL
}
\end{frame}

\begin{frame}{Elige tu lenguaje}
\ctable[
    width = \textwidth,
    doinside = \footnotesize
]{llcl}{
}{                                                                             \FL
    Lenguaje & Versión           & Máx. seg. & Extras                          \ML
    Smalltalk & 3.2.4            & 5         &                                 \NN
    Tcl      & 8.5 con tclsh     & 5         &                                 \NN
    R        & 3.0.2             & 3         &                                 \NN
    RACKET   & 6.1               & 10        &                                 \NN
    RUST     & 1.0               & 5         &                                 \NN
    SWIFT    & 1.2               & 2         & Foundation                      \NN
    PASCAL   & 2.6.2             & 2         &                                 \NN
    Bash     & 4.3.11            & 1         &                                 \NN
    D        & 2.067             & 3         &                                 \LL
}
\end{frame}

\subsection{Problemas}
% TODO: Poner imagenes de programas
\begin{frame}{Emulador / Assembly Simulator}
    \small
    Assembly Language is a low-level programming language that is specific to a particulars computer
    architecture. Each command has a specific structure: \texttt{label COMMAND OPERANDS} \\~
    
    \begin{columns}
    \begin{column}{0.50\textwidth}
        \begin{itemize}
            \item \texttt{PRINT A1}
            \item \texttt{MOVE \#N,A1}
            \item \texttt{MOVE (A1),A2}
            \item \texttt{ADD A1,A2}
            \item \texttt{XOR A1,A2}
            \item \texttt{COMP A1,A2}
            \item \texttt{BEQ label}
            \item \ldots
        \end{itemize}
    \end{column}
    \begin{column}{0.50\textwidth}
        \includegraphics[width=\textwidth]{imgs/q6_result.png}
    \end{column}
    \end{columns}
\end{frame}

\begin{frame}{Elementary Cellular Automaton}
    Elementary Cellular Automaton (ECA) is a discrete modeling technique used in science and engineering to
    study the behavioral patterns that emerge in nature. \\~

    \begin{columns}
    \begin{column}{0.50\textwidth}
        \includegraphics[width=\textwidth]{imgs/eca_rules.png}
    \end{column}
    
    \begin{column}{0.50\textwidth}
        \includegraphics[width=\textwidth]{imgs/eca.jpg}
    \end{column}
    \end{columns}
\end{frame}

\begin{frame}[fragile]{Ingeniería inversa / Run Me}
    This problem seems quite easy: it seems that we are giving you the answer…
    All you have to do is supply the output of our program on the given input. \\~
    
    \textbf{Input}

    A string of characters, ending with a dot ('.'). \\~
    
    \textbf{Output}

    Your program should print the output of an MS-DOS 8086 assembly program
    \texttt{p.com} on the given input.
    
    \begin{lstlisting}
BF 00 04 BE C0 00 56 31 C9 B4 00 CD 16 3C 2E AA
E0 F7 F7 D1 29 D2 89 CD 5B 53 FE 07 75 03 43 EB
F9 BF 00 02 89 F9 89 F8 F3 AA 89 FE AC 89 C3 FE
07 80 FB 2E 75 F6 FE 0F 5E 56 89 E9 AC 89 C3 FE
0F 7C D5 E2 F7 42 5E 56 89 E9 F3 A6 75 CA 5D 92
D4 0A E8 00 00 86 C4 04 30 CD 29 C3
    \end{lstlisting}
\end{frame}

\subsection{Premios}
\begin{frame}{Premios}
    \begin{wideitemize}
        \item \textbf{Primer puesto}: Viaje con todos los gastos pagados a
        un congreso de IEEE.
        
        \item \textbf{Segundo puesto}: iPad Air para cada miembro.
        
        \item \textbf{Tercer puesto}: iPad Mini para cada miembro.
        
        \item \textbf{Cuarto a décimo puesto}: Raspberry Pi para cada miembro.
        
        \item \textbf{Top 100}: Regalo especial. 
    \end{wideitemize}
\end{frame}

\subsection{Una experiencia única}
\begin{frame}{Vive la experiencia}
    \only<1>{\includegraphics[width=\textwidth]{imgs/full_teams.jpg}}
    \only<2>{\includegraphics[width=\textwidth]{imgs/start.jpg}}
    \only<3>{\includegraphics[width=\textwidth]{imgs/starting_work.jpg}}
    \only<4>{
        \begin{columns}\begin{column}{0.50\textwidth}
            \includegraphics[width=\textwidth]{imgs/hard_party.png}
        \end{column}\begin{column}{0.50\textwidth}
            \includegraphics[width=\textwidth]{imgs/hard_beer.jpg}
        \end{column}\end{columns}
    }
    \only<5>{
        \begin{columns}\begin{column}{0.50\textwidth}
            \includegraphics[width=\textwidth]{imgs/como_puedas.jpg}
        \end{column}\begin{column}{0.50\textwidth}
            \includegraphics[width=\textwidth]{imgs/pestanas.jpg}
        \end{column}\end{columns}
    }
    \only<6>{\includegraphics[width=\textwidth]{imgs/eating.jpg}}
    \only<7>{
        \begin{columns}\begin{column}{0.50\textwidth}
            \includegraphics[width=\textwidth]{imgs/duerme0.jpg}
        \end{column}\begin{column}{0.50\textwidth}
            \includegraphics[width=\textwidth]{imgs/duerme1.jpg}
        \end{column}\end{columns}
    }
    \only<8>{\includegraphics[width=\textwidth]{imgs/working.jpg}}
    \only<9>{\includegraphics[width=\textwidth]{imgs/nyan.jpg}}
    \only<10>{
        \movie[width=\textwidth,
	       externalviewer]
	       {\includegraphics[width=\textwidth]{videos/uploading.png}}
	       {videos/uploading.mp4}
    }
\end{frame}